\documentclass[11pt,oneside,letterpaper,twocolumn]{article}

\usepackage{hyperref}
\usepackage{listings}

\begin{document}

\title{
    Role-Based Access Control Filesystem\\
    (RBACFS)
}
\author{
    Scott Linder\\
    \and
    Ryan DePrekel\\
    \and
    Juston Lanyon
}
\date{\today}

\maketitle

\begin{abstract}
This document describes the purpose, implementation, and maintenance of the Role-Based Access Control Filesystem (RBACFS) software. Any assumed
technical knowledge is stated, including links to relevant resources.
Information on obtaining source code, installing prerequisites, and
building/testing the software is provided. Finally the structure of the
codebase, including descriptions of all modules and their inter-dependencies,
is explained.
\end{abstract}

\section{Technologies}

Familiarity with all technologies employed in the development and deployment of
RBACFS is required to get the most out of this document.

\subsection{Git}

All source code and documentation is maintained in the Git version control
system.

\subsection{Language and Compiler}

RBACFS is developed exclusively in the C programming language, using the GNU
Compiler Collection (GCC) compiler. The C standard targetted is {\tt c99} with
GNU extensions, known as {\tt gnu99}.

\subsection{Make}

GNU Make is used to compile both program and documentation source code into
usable programs and documents.

\subsection{FUSE}

Developing a filesystem typically involves modifying kernel code via loadable
kernel modules. Security vulnerabilities in this code effect the security and
stability of the entire system, and all running programs. To avoid this
considerable risk, and to ease development and modification of the RBACFS, it
is written against the Filesystem in Userspace (FUSE) kernel interface. This
allows the filesystem to execute in underprivleged userspace, while still
providing the same interface as a filesystem written as a kernel module.

\subsection{Operating System}

Any operating system which supports a gnu99 compatible compiler, and version 26
of the FUSE kernel interface should be capable of compiling and running RBACFS.
However, the system has be tested exclusively on 32-bit versions of Kali Linux
2016.2 and Ubuntu Linux 14.04. When migrating to a new environment, it is
imperative that the full test suite be run to ensure basic functionality.

\section{Obtaining the Software}

The current central git repository is hosted at
\url{https://github.com/scott-linder/rbacfs} and contains all source code and
documentation.

\section{Building the Software}

\subsection{Prerequisites}

TODO

\subsection{Configure}

There are no configuration steps to build the software.

\subsection{Make}

The software can be built from the {\tt src/} directory with the following
command:
\begin{lstlisting}
make
\end{lstlisting}
which will create a single binary named {\tt rbacfs}.

\subsection{Install}

TODO

\section{Security Considerations}

TODO

\section{System Overview}

TODO

\section{Updates}

TODO

\end{document}
