\documentclass[11pt,oneside,letterpaper]{article}

\usepackage{hyperref}
\usepackage{listings}
\usepackage{graphicx}

\begin{document}

\title{
    Role-Based Access Control Filesystem\\
    (RBACFS)
}
\author{
    Scott Linder\\
    \and
    Ryan DePrekel\\
    \and
    Justin Lanyon
}
\date{\today}

\maketitle

\begin{abstract}
This document describes the purpose, implementation, and maintenance of the
Role-Based Access Control Filesystem (RBACFS) software. Any assumed technical
knowledge is stated, including links to relevant resources. Information on
obtaining source code, installing prerequisites, and building/testing the
software is provided. Finally the structure of the codebase, including
descriptions of all modules and their inter-dependencies, is explained.
\end{abstract}

\section{Technologies}

Familiarity with all technologies employed in the development and deployment of
RBACFS is required to get the most out of this document.

\subsection{Git}

All source code and documentation is maintained in the Git version control
system. Downloads for all major platorms are available from the official
homepage at \url{https://git-scm.com/}.

The current canonical repository for the software is publically hosted at
\url{https://github.com/scott-linder/rbacfs}.

\subsection{Language and Compiler}

RBACFS is developed exclusively in the C programming language, using the GNU
Compiler Collection (GCC) compiler. The C standard targetted is {\tt c99} with
GNU extensions, known as {\tt gnu99}. The version 6.2.1 of {\tt gcc} has been
used during development and testing.

\subsection{Make}

GNU Make is used to compile both program and documentation source code into
usable programs and documents. Most directories in the source tree, save those
containing only static documents, contain one {\tt Makefile}. The version 4.2.1
has been used during development, but any modern version of GNU Make should
suffice.

\subsection{FUSE}

Developing a filesystem typically involves modifying kernel code via loadable
kernel modules. Security vulnerabilities in this code effect the security and
stability of the entire system, and all running programs. To avoid this
considerable risk, and to ease development and modification of RBACFS, it
is written against the Filesystem in Userspace (FUSE) kernel interface. This
allows the filesystem to execute in underprivileged userspace, while still
providing the same interface as a filesystem written as a kernel module.

At least a basic understanding of the FUSE API is required to make changes
to FUSE-specific code in RBACFS, although every attempt has was made
during design and implementation to isolate this code within one module.
Resources describing the use of FUSE are available online, such as
\url{https://lastlog.de/misc/fuse-doc/doc/html/}.

\subsection{Operating System}

Any operating system which supports a gnu99 compatible compiler, and version 26
of the FUSE kernel interface should be capable of compiling and running RBACFS.
However, the system has be tested exclusively on 32-bit versions of Kali Linux
2016.2 and Ubuntu Linux 16.04. When migrating to a new environment, it is
imperative that the full test suite be run to ensure basic functionality.

\section{Obtaining the Software}
\label{sec:obtain}

The current central git repository is hosted at
\url{https://github.com/scott-linder/rbacfs} and contains all source code and
documentation.

\section{Building the Software}
\label{sec:build}

\subsection{Prerequisites}

All development prerequisites have been mentioned above. These include
Linux, {\tt gcc}, FUSE, and {\tt make}.

\subsection{Configure}

There are no separate configuration steps to build the software. As only a
limited set of operating systems are directly supported, the software is
already configured to build and function properly.

\subsection{Make}

The software can be built from the {\tt src/} directory with the following
command:
\begin{lstlisting}
make
\end{lstlisting}
which will create a single binary named {\tt rbacfs}.

\subsection{Install}

The software can be installed from the {\tt src/} directory after it has
been built with the following command:
\begin{lstlisting}
sudo make install
\end{lstlisting}
which will install the binary to {\tt /usr/local/bin}.

The installation prefix may be overridden by setting the {\tt PREFIX}
environment variable (the default being {\tt /usr/local}.) For example,
if the desired installation directory is {\tt /usr/bin}, the following
command may be used:
\begin{lstlisting}
sudo env PREFIX=/usr make install
\end{lstlisting}

\section{Security Considerations}

RBACFS cannot provide true mandatory access control at the VFS level. This
means it cannot actually enforce access control on top of existing filesystems
(e.g. ext4). In order to emulate this behavior, RBACFS ``shadows'' existing
directory hierarchies, applying mandatory control first and then passing
requests through to an existing filesystem.

Requests are thus also subject to discretionary access control checks
based on the effective {\tt uid} and {\tt gid} of the {\tt rbacfs} process.

One way to emulate true mandatory controls is to create a dedicated {\tt
rbacfs} user and group to run the process, change the owner and group of all
files to be protected to {\tt rbacfs}, and set all permissions to {\tt 0700}.

Thus, only the {\tt rbacfs} user (and the {\tt root} user) are able
to access these files, requiring all other users to go through the
shadowed version which implements RBAC.

\section{System Overview}

\subsection{Modules}

The software is divided into internal libraries called ``modules''. The
dependencies of these modules are as shown in figure \ref{fig:modules}.

\begin{figure}[h]
    \centering
    \includegraphics[width=0.5\textwidth]{modules.png}
    \label{fig:modules}
    \caption{Module dependencies}
\end{figure}

\subsubsection{fuse}

The {\tt fuse} module communicates with the FUSE library to actually mount
the filesystem and implement the RBAC policy in the respective filesystem
system calls.

\subsubsection{policy}

The {\tt policy} module converts the parse tree from the {\tt parse} module
into and searchable datastructure to efficiently service queries made by the
{\tt fuse} module.

The two primary interfaces provided are the {\tt user\_role} and {\tt obj\_role\_perms} hashmaps. Given a user, the {\tt user\_role} hashmap
returns a list of roles assigned to that user. Given an object and a role,
the {\tt obj\_role\_perms} hashmap returns a list of permissions the given
role has to the given object.

\subsubsection{parse}

The {\tt parse} module reads the RBAC definition file provided by the user
and lexes and parses it into a list of definition structs. The libraries
used for lexing and parsing are Flex and Bison, respectively.

The grammar of the definition language, in EBNF, is

\begin{lstlisting}
<input> ::= <def-list>
<def-list> ::= <def> | <def-list> <def>
<def> ::= "user: " <id-list> <id-list>
            | "object: " <id-list> <perm-list> <opt-recursive> <path>
<id-list> ::= <id> | <id-list> "," <id>
<perm-list> ::= <perm> | <perm-list> "," <perm>
<opt-recursive> ::= "-r" | ""
<path> ::= /[a-zA-Z0-9_.\-/]*
<id> ::= [a-zA-Z0-9_.\-/]+
<perm> ::= "r" | "w" | "x"
\end{lstlisting}

\subsubsection{list and hashmap}

The final two modules implement simple, generic versions of a linked
list and hashmap, both of which are used repeatedly throughout the
other modules.

\subsection{Module Structure}

Each module contains a {\tt Makefile}, {\tt lib.c}, {\tt lib.h}, and any other
source files required to build the library. Each {\tt Makefile} contains a rule
for building {\tt lib.a}, a static library which exports at least those
functions found in {\tt lib.h}. Thus, modules may depend on each other's {\tt
lib.a} and discover function prototypes in each other's {\tt lib.h}.

\subsection{Control Flow}

The program entry point is defined in {\tt rbacfs.c} in the root of {\tt src/}. The last argument is used as a filename for the {\tt parse} module to
generate a list of definitions, and then definitions are passed to the
{\tt policy} module. Finally, the remaining commandline arguments, along
with the policy structure, are sent to the {\tt fuse} module, which
mounts the filesystem and begins listening for filesystem accesses.

\section{Updates}

Updates can be applied simply by going through the same set of steps found
in sections \ref{sec:obtain} and \ref{sec:build} above. Re-installation is
equivalent to updating.

\end{document}
